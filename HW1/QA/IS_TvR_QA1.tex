\documentclass{article}

\usepackage{fancyhdr}
\usepackage{extramarks}
\usepackage{xcolor}
\usepackage[left=1in]{geometry}
\usepackage[shortlabels]{enumitem}


%%%%%%%%%%%%%%%%%%%%%%%%%%%%%%%%%%%%%%%%%%%%%%%%%%%%%%%%%%%%%%%%%%%%%%%%%
%%%%%%%%%%%%%%%%%%%%%%%%%%%%% PDF COLOR %%%%%%%%%%%%%%%%%%%%%%%%%%%%%%%%%
%%%%%%%%%%%%%%%%%%%%%%%%%%%%%%%%%%%%%%%%%%%%%%%%%%%%%%%%%%%%%%%%%%%%%%%%%
% \usepackage{xcolor} \pagecolor[rgb]{0,0,0} \color[rgb]{0.9,0.9,0.9}
%%%%%%%%%%%%%%%%%%%%%%%%% SUPPRESS UNDERFULL HBOX %%%%%%%%%%%%%%%%%%%%%%%
\hbadness = 20000
%%%%%%%%%%%%%%%%%%%%%%%%%%%%%%%%%%%%%%%%%%%%%%%%%%%%%%%%%%%%%%%%%%%%%%%%%
%
% Basic Document Settings 
%

\topmargin=-0.75in
\textwidth=6.5in
\textheight=9.0in
\headsep=0.25in

\linespread{1.1}
\fancypagestyle{plain}{}
\pagestyle{fancy}
\fancyhf{}

\chead{\hmwkClass: \hmwkTitle}
\lfoot{\lastxmark}
\cfoot{\thepage}

\renewcommand\headrulewidth{0.4pt}
\renewcommand\footrulewidth{0.4pt}

\setlength\parindent{0pt}

\newcommand{\bd}{\textbf}
\newcommand{\hmwkTitle}{Homework\ \#2}
\newcommand{\hmwkClass}{Information Security - Theory vs Reality}


% use for comment block, surround block with "\/*" and at the end "*/" 
\long\def\/*#1*/{}


%%%%%%%%%%%%%%%%%%%%%%%%%%%%%%%%%%%%%%%%%%%%%%%%%%%%%%%%%%%%%%%%%%%%%%%%%
\input{aux_tex/personal_info.tex}
%%%%%%%%%%%%%%%%%%%%%%%%%%%%%%%%%%%%%%%%%%%%%%%%%%%%%%%%%%%%%%%%%%%%%%%%%

%
% Title 
%

\title{
    \textmd{\bd{\hmwkClass:\ \\ \hmwkTitle}}\\
}
\author{\hmwkAuthorName}

\begin{document}

\maketitle

\bd{Question.} \\
Similar to what was described in class, we assume an attacker is able to
measure the exact run time of the decryption function.
What is any information can  such an attacker learn about decrypted valid messages?
What is any information can such an attacker learn about invalid decrypted messages (e.g.,
messages with invalid padding)? \\ \\

\bd{Answer.}

\underline{CBC-HMAC} \\
For decrypted valid messages: \\
Since the run time of the "\_\_decrypt" and "\_\_auth" methods is linear
in the number of blocks of the message and the run time of "\_\_strip\_padding"
is dependant on the padding length, the attacker can calculate the exact length
of the message from the number of blocks in the message (which we know from the ciphertext)
and the required padding length.

Furthermore, if the attacker can modify the ciphertext, he can use his knowledge of the run time
to know if the padding is correct after changing the ciphertext. Using that way he would be able to find
the plaintext byte by byte (as we will see in the Lucky 13 attack)

For decrypted invalid messages: \\
The attacker would know if the padding is invalid or if the HMAC is incorrect,
since he will be able to measure whether the "\_\_auth" method was called or not. \\


\underline{PKCS} \\
Notice that the "decrypt" method takes a constant amount of time since the length of the encrypted data is always $k$
hence the attacker would mainly be able to measure the run time of the "parse" method.

Hence for decrypted valid messages the attacker would be able to know the size of the padding and from
that calculate the length of the message.

Furthermore, if the attacker can modify the ciphertext, he can use his knowledge of the run time
to know if the padding is correct after changing the ciphertext. Using this the attack can perform
a Bleichenbacher attack and find the plaintext.

For decrypted invalid messages the padding is incorrect, so the attacker would not be able to calculate
the length of the message from the "parse" method.


\end{document}